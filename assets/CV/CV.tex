\documentclass[11pt]{resume} %Sets the default text size to 11pt and class to article.
%------------------------Dimensions--------------------------------------------
\topmargin=0.0in %length of margin at the top of the page (1 inch added by default)
\oddsidemargin=0.0in %length of margin on sides for odd pages
\evensidemargin=0in %length of margin on sides for even pages
\textwidth=6.5in %How wide you want your text to be
\marginparwidth=0.5in
\headheight=0pt %1in margins at top and bottom (1 inch is added to this value by default)
\headsep=0pt %Increase to increase white space in between headers and the top of the page
\textheight=9.0in %How tall the text body is allowed to be on each page
\usepackage[utf8]{inputenc}
\usepackage{amsmath}
\usepackage{amsfonts}
\usepackage{amssymb}
\usepackage{changepage}
\usepackage{color}
\usepackage{hyperref}
\usepackage{enumitem}
\definecolor{blue}{cmyk}{1.0, 0.35, 0.0, 0.4}
\parindent 0pt

\usepackage[style=numeric, backend=biber, sorting=none, maxbibnames=99, doi=false]{biblatex}
\addbibresource{bibliography.bib}

\newcommand{\lt}{\ensuremath <}
\newcommand{\gt}{\ensuremath >}

\def\aap{{A\&A}}
\def\mnras{{MNRAS}}
\def\apj{{ApJ}}
\def\apjl{{ApJL}}
\def\apjs{{ApJS}}

\AtDataInput{%
  \csnumgdef{entrycount:\therefsection}{%
    \csuse{entrycount:\therefsection}+1}}

\DeclareFieldFormat{labelnumber}{\mkbibdesc{#1}}
\newrobustcmd*{\mkbibdesc}[1]{%
  \number\numexpr\csuse{entrycount:\therefsection}+1-#1\relax}

\defbibenvironment{bibliography}
{\list
  {\printtext[labelnumberwidth]{%
      \printfield{labelprefix}%
      \printfield{labelnumber}}}
  {\setlength{\labelwidth}{\labelnumberwidth}%
    \setlength{\leftmargin}{\labelwidth}%
    \setlength{\labelsep}{\biblabelsep}%
    \addtolength{\leftmargin}{\labelsep}%
    \addtolength{\leftmargin}{\leftskip}%
    \setlength{\itemsep}{\bibitemsep}%
    \setlength{\parsep}{\bibparsep}}%
  \renewcommand*{\makelabel}[1]{\hss##1}}
{\endlist}
{\item}

\begin{document}

\centerline{\color{blue} \LARGE \bf Johannes Ulf Lange}
\centerline{Yale University, Astronomy Department}
\centerline{52 Hillhouse Avenue, New Haven, CT 06511}
\centerline{johannesulf.lange@yale.edu}
\bigskip

\begin{rSection}{Research Interests}
  \begin{adjustwidth}{0.5cm}{}
    Galaxy-Halo Connection, Galaxy Formation Theory, Cosmology, Large-Scale Structure, Statistical Methods and Machine Learning
  \end{adjustwidth}
\end{rSection}


\begin{rSection}{Education}
  \begin{adjustwidth}{0.5cm}{}
    \begin{rSubsection}{Yale University}{08/2014 -- present}{M.Sc., M.Phil, Ph.D. Candidate in Astronomy}{}\end{rSubsection}

    \begin{rSubsection}{Ruprecht-Karls-Universität Heidelberg}{09/2012 -- 08/2014}{Master of Science in Physics}{}\end{rSubsection}

    \begin{rSubsection}{Freie Universität Berlin}{10/2009 -- 08/2012}{Bachelor of Science in Physics}{}\end{rSubsection}
  \end{adjustwidth}
\end{rSection}

\begin{rSection}{Research}
  \begin{adjustwidth}{0.5cm}{}
    \begin{rSubsection}{Yale University}{08/2014 -- present}{Graduate Student Researcher\\The Galaxy-Halo Connection\\Advisor: Frank van den Bosch}{}\end{rSubsection}
    \begin{rSubsection}{Shanghai Jiao Tong University}{05/2018 -- 07/2018}{Visiting Scholar\\The non-linear Galaxy Clustering and Lensing in BOSS\\Advisor: Xiaohu Yang}{}\end{rSubsection}
    \begin{rSubsection}{Kavli Institute for Theoretical Physics}{01/2017 -- 07/2017}{KITP Graduate Fellow\\The Galaxy-Halo Connection}{}\end{rSubsection}
    \begin{rSubsection}{The Chinese University of Hong Kong}{08/2013 -- 07/2014}{Postgraduate Research Exchange Program\\Dark Matter Annihilation and the Cosmic Gamma-Ray Background\\Advisor: Ming Chung Chu}{}\end{rSubsection}
    \begin{rSubsection}{German Electron Synchrotron (DESY)}{04/2012 -- 07/2012}{Bachelor Thesis\\The average GeV-band Emission from Gamma-Ray Bursts\\Advisor: Martin Pohl}{}\end{rSubsection}
    \begin{rSubsection}{University of California, Santa Barbara}{09/2011 -- 03/2012}{Undergraduate Research\\Search for a Higgs Boson in the Decay Channel $H \rightarrow Z\gamma$\\Advisor: Claudio Campagnari}{}\end{rSubsection}
  \end{adjustwidth}
\end{rSection}

\begin{rSection}{Publications}
  \setlength{\leftskip}{0.5cm}
  \nocite{10, 9, 8, 7, 6, 5, 4, 3, 2, 1}
  \printbibliography[heading=none]
\end{rSection}


\begin{rSection}{Talks}
  \begin{adjustwidth}{0.5cm}{}
    \textbf{BCCP Seminar} (invited)\hfill 09/2018\\
    University of California, Berkeley

    \textbf{CosmoClub}\hfill 09/2018\\
    University of California, Santa Cruz

    \textbf{SUGAR-RUSH Conference}\hfill 06/2018\\
    Shanghai Jiao Tong University

    \textbf{KIPAC Tea Talk}\hfill 03/2018\\
    SLAC National Accelerator Laboratory

    \textbf{Astronomy Seminar}\hfill 03/2018\\
    University of California, Riverside

    \textbf{Galaxy Coffee}\hfill 01/2018\\
    Max Planck Institute for Astronomy

    \textbf{The Galaxy-Halo Connection Across Cosmic Time} (invited)\hfill 07/2017\\
    Kavli Institute for Theoretical Physics

    \textbf{Quantifying and Understanding the Galaxy-Halo Connection}\hfill 05/2017\\
    Kavli Institute for Theoretical Physics

    \textbf{Astroparticle Physics Seminar}\hfill 07/2014\\
    German Electron Synchrotron (DESY)

    \textbf{Conference of the Physical Society of Hong Kong}\hfill 06/2014\\
    Hong Kong Baptist University

    \textbf{IAS Workshop on New Perspectives in Cosmology}\hfill 05/2014\\
    Hong Kong University of Science and Technology
  \end{adjustwidth}
\end{rSection}


\begin{rSection}{Teaching}
  \begin{itemize}[leftmargin=1.0cm, topsep=0pt,itemsep=0pt,partopsep=0pt, parsep=0pt]
    \item Astrostatistics and Data Mining, Lab Leader, Yale University, Spring 2018
    \item Introduction to Astronomical Observing, Lab TA, Yale University, Fall 2017
    \item Astrostatistics and Data Mining, Lab Leader, Yale University, Spring 2016
    \item Introduction to Cosmology, Section Leader, Yale University, Fall 2015
    \item Gravity, Astrophysics, and Cosmology, Grader, Yale University, Spring 2015
    \item Introduction to Astronomical Observing, Lab TA, Yale University, Fall 2014
  \end{itemize}
\end{rSection}

\begin{rSection}{Outreach}
  \begin{itemize}[leftmargin=1.0cm, topsep=0pt,itemsep=0pt,partopsep=0pt, parsep=0pt]
    \item Talk at Leitner Family Observatory \& Planetarium, New Haven, CT, February 2018
    \item Talk at Open Labs Science Cafe Talk, Yale University, New Haven, CT, October 2017
    \item Member of Open Labs, Yale University, New Haven, CT, since 2016
    \item Tutor at New Haven Reads, New Haven, CT, 2015 - 2018
    \item Member of UCSB Physics Circus, UC Santa Barbara, Santa Barbara, CA, 2012
  \end{itemize}
\end{rSection}

\begin{rSection}{Skills}
  \begin{itemize}[leftmargin=1.0cm, topsep=0pt,itemsep=0pt,partopsep=0pt, parsep=0pt]
    \item Programming Languages – C/C++, Python, Cython, Java
    \item Scientific Applications – NumPy, SciPy, matplotlib, LaTeX, git
    \item Languages – German (native), English (fluent), Chinese (basic)
  \end{itemize}
\end{rSection}

\begin{rSection}{Honors and Awards}
  \begin{itemize}[leftmargin=1.0cm, topsep=0pt,itemsep=0pt,partopsep=0pt, parsep=0pt]
    \item KITP Graduate Fellowship Program
    \item Henry A. Smith Fellowship, Yale University
    \item DAAD (German Academic Exchange Service) Scholarship
    \item Deutschlandstipendium National Scholarship Program
    \item Ernst Reuter Scholarship, Free University of Berlin
    \item Dean's List, University of California, Santa Barbara
  \end{itemize}
\end{rSection}

\end{document}
